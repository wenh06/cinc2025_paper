\section{Methods}
\label{sec:methods}

% NOT finished

\subsection{Dataset Preparation}
\label{subsec:data}

% almost finished

We used three ECG datasets for model training, with substantial differences in sample size, Chagas prevalence, and label provenance as summarized in Table~\ref{tab:dataset_stats}.

\begin{table}[!htp]
\centering
\begin{tabular}{rlll}
\toprule
Dataset & \multicolumn{1}{c}{Size} & \multicolumn{1}{c}{Chagas rate} & Label provenance \\
\midrule
SaMi-Trop & 1\,631   & 100 & expert-confirmed \\
CODE-15\% & 345\,779 & 1.795   & self-reported \\
PTB-XL    & 21\,799  & 0   & N/A \\
\bottomrule
\end{tabular}
\caption{Dataset statistics and label provenance. Chagas rate is the proportion of recordings labeled positive in each dataset. N/A indicates that confirmed Chagas cases are not expected (non-endemic population).}
\label{tab:dataset_stats}
\end{table}

All ECGs were uniformly resampled to 400 Hz, bandpass filtered (0.5–45 Hz), and z-score normalized to zero mean and unit variance computed as in Eq.~\ref{eq:z_score}:
\begin{equation}
\label{eq:z_score}
\tilde{\mathbf{x}} = \frac{\mathbf{x} - \boldsymbol{\mu}_x}{\boldsymbol{\sigma}_x}
\end{equation}
where $\mathbf{x}$ is the original ECG signal, $\boldsymbol{\mu}_x$ and $\boldsymbol{\sigma}_x$ are the mean and standard deviation of $\mathbf{x}$, respectively.

To address label reliability, we applied stratified label smoothing \cite{inceptionv2v3}: expert-confirmed positives (SaMi-Trop) use hard labels (smoothing factor $0.0$), self-reported positives (CODE-15\%) use moderate smoothing ($0.6$), and negatives use higher smoothing ($0.2$). This reflects greater trust in expert-confirmed labels and accounts for possible noise in other groups. Label smoothing is computed according to Eq.~\ref{eq:label_smoothing}:
\begin{equation}
\label{eq:label_smoothing}
\tilde{\mathbf{y}} = (1 - \varepsilon) \cdot \mathbf{y} + \frac{\varepsilon}{K} \cdot \mathbf{1},
\end{equation}
where $\mathbf{y}$ is the one-hot ground-truth label ($[0, 1]$ for positives, $[1, 0]$ for negatives), $\varepsilon$ is the smoothing factor, and $\tilde{\mathbf{y}}$ is the smoothed label. It should be noted that the smoothing factor $\varepsilon$ varies across samples from different sources.

Severe class imbalance was mitigated by upsampling positives during training: positive samples from CODE-15\% were upsampled by a factor of 3, and those from SaMi-Trop by 7. No upsampling was applied to PTB-XL, which contains no positives. Smoothed labels and upsampling strategies for each dataset are summarized in Table~\ref{tab:augmentation}.

\begin{table}[!htp]
\centering
\begin{tabular}{lcccc}
\toprule
\multirow{2}{*}{Dataset} & Upsampling & \multicolumn{2}{c}{Smoothed labels} \\ \cmidrule(lr){3-4}
& factor & negative & positive \\
\midrule
SaMi-Trop & $7$ & N/A           & $[0,\,1]$     \\
CODE-15\%   & $3$ & $[0.7,\,0.3]$ & $[0.3,\,0.7]$ \\
PTB-XL    & $1$ & $[0.9,\,0.1]$ & N/A           \\
\bottomrule
\end{tabular}
\caption{Smoothed labels and upsampling strategies for each dataset. Smoothed labels are computed from Eq.~\ref{eq:label_smoothing}.}
\label{tab:augmentation}
\end{table}


\subsection{xxx}

% NOT finished


\cite{torch_ecg_paper}
