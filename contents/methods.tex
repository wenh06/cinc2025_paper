\section{Methods}
\label{sec:methods}

% almost finished

\subsection{Datasets and Preprocessing}
\label{subsec:data}

% almost finished

We used three ECG datasets for model training, with substantial differences in sample size, Chagas prevalence, and label provenance as summarized in Table~\ref{tab:dataset_stats}.

\setlength{\tabcolsep}{4pt} % Default value: 6pt
\begin{table}[!htp]
\centering
\begin{tabular}{rlll}
\toprule
Dataset & \multicolumn{1}{c}{Size} & \multicolumn{1}{c}{Chagas rate} & Label provenance \\
\midrule
SaMi-Trop   & 1\,631     & 100 \%    & expert-confirmed \\
CODE-15\%   & 345\,779   & 1.795 \%  & self-reported \\
PTB-XL      & 21\,799    & 0 \%      & N/A \\
\bottomrule
\end{tabular}
\caption{Dataset statistics and label provenance. Chagas rate is the proportion of recordings labeled positive in each dataset. N/A indicates that confirmed Chagas cases are not expected (non-endemic population).}
\label{tab:dataset_stats}
\end{table}
\setlength{\tabcolsep}{6pt} % Default value: 6pt

All ECGs were uniformly resampled to 400 Hz, bandpass filtered (0.5–45 Hz), and z-score normalized to zero mean and unit variance computed as in Eq.~\ref{eq:z_score}:
\begin{equation}
\label{eq:z_score}
\tilde{\mathbf{x}} = \frac{\mathbf{x} - \boldsymbol{\mu}_{\mathbf{x}}}{\boldsymbol{\sigma}_{\mathbf{x}}}
\end{equation}
where $\mathbf{x}$ is the original ECG signal, $\boldsymbol{\mu}_{\mathbf{x}}$ and $\boldsymbol{\sigma}_{\mathbf{x}}$ are the mean and standard deviation of $\mathbf{x}$, respectively. We excluded ECGs shorter than 1200 samples to ensure inputs contain enough cardiac cycles for stable model analysis.


\begin{figure*}[!t]
\centering
\adjustbox{width=0.70\linewidth}{
\begin{tikzpicture}[
    % font=\small,
    line width=0.75pt,
    >={Latex[length=3mm]},
    node distance=6.4mm,
    every node/.style={inner sep=2pt},
    stem/.style={draw, rounded corners=2pt, minimum width=30mm, minimum height=10mm,
                 align=center, fill=blue!10},
    block/.style={draw, rounded corners=2pt, minimum width=30mm, minimum height=12mm,
                  align=center, fill=orange!12},
    se/.style={draw, rounded corners=2pt, minimum width=30mm, minimum height=10mm,
               align=center, fill=green!12},
    pool/.style={draw, rounded corners=2pt, minimum width=30mm, minimum height=10mm,
                 align=center, fill=cyan!10},
    head/.style={draw, rounded corners=2pt, minimum width=30mm, minimum height=10mm,
                 align=center, fill=purple!10},
    output/.style={draw, rounded corners=2pt, minimum width=24mm, minimum height=10mm,
                align=center, fill=red!10},
    arrow/.style={-{Latex[length=3mm]}, thick},
    darr/.style={-{Latex[length=2.4mm]}, thick},
    dashedarrow/.style={-{Latex[length=2.6mm]}, thick, dashed},
    ann/.style={inner sep=1pt, align=center},
    layer/.style={draw, rounded corners=2pt, minimum width=18mm, minimum height=6.2mm,
                  align=center, fill=blue!6},
    bnact/.style={draw, rounded corners=2pt, minimum width=18mm, minimum height=6.2mm,
                  align=center, fill=violet!10},
    act/.style={draw, rounded corners=2pt, minimum width=18mm, minimum height=6.2mm,
                  align=center, fill=red!10},
    drop/.style={draw, rounded corners=2pt, minimum width=18mm, minimum height=6.2mm,
                  align=center, fill=gray!15, dashed},
    sum/.style={draw, circle, inner sep=1.6pt, fill=white},
    % brace/.style={decorate, decoration={brace, amplitude=6pt}},
    detailbox/.style={draw, rounded corners=3pt, inner sep=6pt, fill=orange!3}
]

% ================= Left (overall vertical path) =================
\node[stem] (input) {Input \\ (12$\times$L)};
\node[stem, below=of input] (stem) {Stem \\ Conv ks=15 \\ BN + ReLU \\ 64 Ch};

% Compressed single bottleneck block (represents 4 stacked blocks)
\node[block, below=of stem] (bottleneck) {$4 \times$ Bottleneck Block \\ ks: 1--15--1 (stride=4) \\ Ch: $512 \to 768 \to 1024 \to 1280$};

% Global SE
\node[se, below=of bottleneck] (gse) {Global SE \\ reduction=8};

% Global pooling
\node[pool, below=of gse] (gap) {Global MaxPool};

% MLP Head
\node[head, below=of gap] (mlp) {MLP Head \\ (1280 $\rightarrow$ C=2)};

% Output
\node[output, below=of mlp] (out) {Output Logits};

% Arrows (overall path)
\draw[arrow] (input) -- (stem);
\draw[arrow] (stem) -- (bottleneck);
\draw[arrow] (bottleneck) -- (gse);
\draw[arrow] (gse) -- (gap);
\draw[arrow] (gap) -- (mlp);
\draw[arrow] (mlp) -- (out);

% ================= Right (expanded single bottleneck internals) =================
% Anchor for expansion (to the right of overall bottleneck block)
\coordinate (expandAnchor) at ($(input.east)+(45mm,0)$);

\node[layer, below=0mm of expandAnchor] (c1) {Conv 1$\times$1};
\node[below right=-1mm and 8mm of c1.south] {reduce};
\node[bnact, below=5mm of c1] (bn1) {BN};
\node[act, below=5mm of bn1] (a1) {ReLU};

\node[layer, below=5mm of a1] (c2) {Conv 15\\(stride 4)};
\node[bnact, below=5mm of c2] (bn2) {BN};
\node[act, below=5mm of bn2] (a2) {ReLU};
% \node[drop, below=5mm of a2] (drp) {Dropout 0.2};

\node[layer, below=5mm of a2] (c3) {Conv 1$\times$1};
\node[below right=-1mm and 8mm of c3.south] {expand};
\node[bnact, below=5mm of c3] (bn3) {BN};

\node[sum, below=5mm of bn3] (add) {+};
\node[act, below=5mm of add] (outact) {ReLU (output)};

% Projection/skip branch (since stride=4 or channel change)
\node[layer, right=10mm of c2, yshift=-7mm, minimum width=18mm] (proj) {Proj 1$\times$1 \\ (stride 4)};
\node[bnact, below=5mm of proj, minimum width=18mm] (projbn) {BN};

% Connections internal
\draw[darr] (c1) -- (bn1);
\draw[darr] (bn1) -- (a1);
\draw[darr] (a1) -- (c2);
\draw[darr] (c2) -- (bn2);
\draw[darr] (bn2) -- (a2);
% \draw[darr] (a2) -- (drp);
\draw[darr] (a2) -- (c3);
\draw[darr] (c3) -- (bn3);
\draw[darr] (bn3) -- (add);
\draw[darr] (add) -- (outact);

% Skip path
\draw[darr] (c1.east) -| ($(proj.north)+(0,0mm)$);
\draw[darr] (proj) -- (projbn);
\draw[darr] (projbn.south) |- (add.east);

\begin{scope}[on background layer]
\node[detailbox, fit=(c1) (bn1) (a1) (c2) (bn2) (a2) (c3) (bn3) (proj) (projbn) (add) (outact)] (detail) {};
\end{scope}

\draw[dashed, line width=0.9pt] (bottleneck.north east) -- (detail.north west);
\draw[dashed, line width=0.9pt] (bottleneck.south east) -- (detail.south west);

\end{tikzpicture}

}
\caption{Model architecture. Left: overall network: a stem (Conv1d, kernel size 15, 64 output channels, Batch Normalization, ReLU) followed by four bottleneck residual blocks, a global SE module, global max pooling, and an MLP head producing $C = 2$ logits. Channel widths shown ($512 \to 768 \to 1024 \to 1280$) are the expanded channels.
Right: internal bottleneck structure. The middle convolution of kernel size 15 uses a stride of 4 for temporal downsampling; kernel size 1 convolutions reduce and then expand channels, and a projection convolution (kernel size 1, stride 4) aligns resolution and width for the residual path. For clarity, dropout layers present in the implementation are omitted. Abbreviations: ks kernel size; Ch channels; BN Batch Normalization; SE squeeze-and-excitation; MLP multi-layer perceptron.}
\label{fig:model}
\end{figure*}


\subsection{Reliability-Aware Hierarchical Supervision}
\label{subsec:reliability}

% almost finished

We introduce a hierarchical supervision scheme that encodes source reliability through stratified label smoothing and adaptive upsampling. Three reliability levels are defined: (1) expert-confirmed positives (SaMi-Trop, maximal trust), (2) self-reported samples (CODE-15\%, both positives and negatives, higher uncertainty), and (3) non-endemic negatives (PTB-XL, very low true prevalence but still mildly regularized). Given a one-hot label $\mathbf{y}$ ($[0, 1]$ for positives, $[1, 0]$ for negatives) and number of classes $C = 2,$ the smoothed target is computed as in Eq.~\ref{eq:label_smoothing}:
\begin{equation}
\label{eq:label_smoothing}
\tilde{\mathbf{y}} = (1 - \varepsilon) \cdot {\mathbf{y}} + \frac{\varepsilon}{C} \cdot {\mathbf{1}},
\end{equation}
where $\varepsilon$ is the smoothing factor which depends on the reliability level: 0.0 (SaMi positives), 0.6 (CODE-15\% positives \& negatives), 0.2 (PTB-XL negatives). This attenuates overconfident gradients for noisier or potentially misreported labels while preserving sharp supervision on expert-confirmed cases.

Severe class imbalance was mitigated by upsampling positives during training: positive samples from CODE-15\% were upsampled by a factor of 3, and those from SaMi-Trop by 12. No upsampling was applied to PTB-XL, which contains no positives. We chose these factors after reviewing hidden validation scores from multiple submissions, as shown in Table~\ref{tab:upsampling_schemes}.
\begin{table}[!htbp]
\centering
\begin{tabular}{ccc}
\toprule
\multicolumn{2}{c}{Upsample factor} & \multirow{2}{*}{Challenge score} \\ \cmidrule(lr){1-2}
CODE-15\% & SaMi-Trop               &                                  \\ \midrule
-         & -                       & 0.239$^\dagger$                        \\
3         & 7                       & 0.212                            \\
3         & 12                      & \textbf{0.245}                   \\
10        & 120                     & 0.210                            \\
6         & 36                      & 0.221                            \\
\bottomrule
\end{tabular}
\caption{Representative upsampling schemes and corresponding Challenge scores on the hidden validation set. The model and training strategies used were the same. “-” indicates no upsampling.\\
$^\dagger$ obtained during the unofficial phase.
}
\label{tab:upsampling_schemes}
\end{table}


Smoothed labels and upsampling strategies for each dataset are summarized in Table~\ref{tab:augmentation}.


\begin{table}[!htp]
\centering
\begin{tabular}{lcccc}
\toprule
\multirow{2}{*}{Dataset} & Upsampling & \multicolumn{2}{c}{Smoothed labels} \\ \cmidrule(lr){3-4}
& factor & negative & positive \\
\midrule
SaMi-Trop & $12$ & N/A          & $[0,\,1]$     \\
CODE-15\% & $3$ & $[0.7,\,0.3]$ & $[0.3,\,0.7]$ \\
PTB-XL    & $1$ & $[0.9,\,0.1]$ & N/A           \\
\bottomrule
\end{tabular}
\caption{Smoothed labels (computed from Eq.~\ref{eq:label_smoothing}) and upsampling strategies for each dataset.}
\label{tab:augmentation}
\end{table}


\subsection{Model Architecture}
\label{subsec:model}

% almost finished

We build upon the 1D ResNet ECG classifier of Ribeiro et al.~\cite{ribeiro2020automatic} and introduce three modifications.

\textbf{(1) Bottleneck residual blocks.} We replace basic ResNet blocks with bottleneck blocks of kernel sizes $1$–$15$–$1$ (pointwise--temporal--pointwise). The middle temporal convolution applies a stride of 4 for downsampling; the two convolutions (kernel size 1) first reduce the number of channels and then expand them with an expansion factor of 4. A projection convolution (kernel size 1, stride 4) is used in the residual branch whenever temporal resolution or channel width changes. Across the four blocks, the reduced (bottleneck) channel widths are $128 \to 192 \to 256 \to 320$, yielding expanded output widths $512 \to 768 \to 1024 \to 1280$.

\textbf{(2) Global squeeze-and-excitation (SE).} After the final bottleneck block, a single global SE module (reduction ratio 8) \cite{hu2018senet} performs temporal average pooling to a channel descriptor, applies a two-layer bottleneck multi-layer perceptron (MLP) $1280 \to 160 \to 1280$ with ReLU and sigmoid gating, and rescales the feature map channel-wise.

\textbf{(3) Global pooling head for variable input length.} Instead of flattening a fixed-length feature map as in the original baseline, we apply global max pooling over the remaining temporal dimension, yielding a 1280-dimensional vector irrespective of input length $L$. This vector is fed to a lightweight two-layer classification MLP: a hidden fully connected layer ($1280 \to 1024$) with non-linear activation and dropout (rate 0.2), followed by a final linear layer ($1024 \to 2$) producing class logits.

\textbf{Stem and regularization.} A stem Conv1d (kernel size 15, stride 1, 64 channels) with BatchNorm and ReLU precedes the bottleneck stack. Within each bottleneck block, we apply BatchNorm+ReLU after the first two convolutions and dropout (rate 0.2) after each of those activations. All convolutions use “same” padding to preserve temporal length before downsampling operations.

% Overall, the changes primarily enable variable-length inference and add global channel reweighting with minimal structural overhead.
The overall model architecture is illustrated in Fig.~\ref{fig:model}.


\subsection{Training and Implementation Setups}
\label{subsec:train}

% almost finished

We employed asymmetric loss (ASL) \cite{ridnik2021asymmetric_loss} to complement the reliability-aware label smoothing strategy, jointly addressing the challenges of severe class imbalance. Let ${\mathbf{z}} = (z_0, z_1)$ denote the logits and $p=\operatorname{softmax}({\mathbf{z}})_1$ the predicted probability of the positive class. The ASL is defined in Eq.~\ref{eq:asl} with separate focusing parameters for positives and negatives and a clipped negative probability term:
\begin{equation}
\label{eq:asl}
\begin{multlined}
L = -y \cdot (1-p)^{\gamma_{+}} \log(p) \\
\phantom{L = } - (1-y) \cdot (p_m)^{\gamma_{-}} \log(1-p_m),
\end{multlined}
\end{equation}
where $y$ is the (smoothed) positive-class target probability, $p_m = \max(p - m, 0),$ $(\gamma_{+},\gamma_{-})=(1,4)$ and margin $m=0.05$. We train for 30 epochs with batch size 128 using the AdamW optimizer (initial learning rate $1\times10^{-4}$, peak $6\times10^{-4}$ under a OneCycle scheduler, weight decay $1\times10^{-2}$). Early stopping (patience 10 epochs, monitored on a fixed 20\% internal hold-out subset) selects the final model via the Challenge metric. Each training segment is a uniform random crop (or center padding if shorter) of length 4096 samples. The full implementation, including model construction, data pipeline, and optimization utilities, is based on the \texttt{torch-ECG} framework \cite{torch_ecg_paper}.
