% -*- Mode:TeX -*-

\documentclass{cinc-abstract}

\usepackage{graphicx}
\usepackage{xcolor}
\usepackage{relsize}
\usepackage{pifont}
\usepackage{currfile}

\newcommand\wordcount{\input{|"texcount -inc -sum -0 -utf8 -ch -template={SUM} \currfilepath"}}


\begin{document}

% The title is set in 14-point Helvetica bold
% \title{Hierarchical Learning with Clinical Expertise Integration for Chagas Detection from ECG under Data Scarcity Constraints}
\title{Hierarchical Learning for Chagas Detection from ECG under Expert-Label Scarcity}

% The rest of the title block is set in 12 points Helvetica
\author {Hao Wen, Jingsu Kang\\ % First name, initials and surnames, no ``and''
\ \\ % leave an empty line between authors and affiliation
China Agricultural University\\  % gives affiliation of the first author only
Beijing, China} % city, [state or province,] country only

\maketitle

%%%%%%%%%%%%%%%%%%%%%%%%%%%%%%%%%%%%%%%%%%%%%%%%%%%%%%%%%%
% NOTE: The body of the abstract (exclusive of the title, authors, and authors’ affiliations) can be up to 300 words at most
%%%%%%%%%%%%%%%%%%%%%%%%%%%%%%%%%%%%%%%%%%%%%%%%%%%%%%%%%%


% Aim: This paper tackles automated Chagas cardiomyopathy screening from electrocardiograms (ECGs), as the George B. Moody PhysioNet Challenge 2025 proposes.
Aim: This paper introduces a reliability-aware hierarchical learning framework for automated Chagas cardiomyopathy screening from electrocardiograms (ECGs), serving as the ``Revenger'' team's solution to the George B. Moody PhysioNet Challenge 2025.

Methods: ECGs were preprocessed through resampling to 400 Hz, 0.5-45 Hz Butterworth bandpass filtering, and z-score normalization to zero mean and unit variance. We implemented a convolutional neural network (CNN) based on the ResNet architecture with integrated squeeze-and-excitation (SE) modules for binary prediction (Chagas negative/positive). The public training data are highly imbalanced, with only approximately 2\% positive samples and an even larger scarcity (0.2\%) of expert-labeled samples, where the rest are patient self-reported. We employed a hierarchical upsampling and label smoothing scheme that prioritizes expert-validated samples over patient-reported ones. Expert-validated samples received a higher upsampling rate and a lower label smoothing factor (indicating higher reliability). Our model was optimized using asymmetric loss to further penalize false negatives, and with Adam optimizer and \texttt{OneCycle} learning rate scheduler for rapid convergence. A demographic feature-based stratified split was conducted to reserve 20\% of the data for model selection based on the challenge score: maximizing true positive rate while keeping positive predictions under 5\% prevalence.

Results: Our approach received a challenge score of 0.290 (Ranking: 76) on the hidden validation set. In our internal evaluation, the highest score was 0.456.

Conclusion: The proposed method establishes an effective foundation for automated Chagas screening through ECG analysis in resource-limited settings, yet presents clear opportunities for refinement to enhance detection performance and clinical applicability.


% \ifhandout
% % do nothing
% \else
% \begin{center}
% {\larger[2]\color{red} \ding{43}\ding{43}  Totally \wordcount words \reflectbox{\ding{43}\ding{43}}}
% \end{center}
% \fi


\end{document}
