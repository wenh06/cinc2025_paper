% -*- Mode:TeX -*-

\documentclass{cinc-abstract}

\usepackage{graphicx}
\usepackage{xcolor}
\usepackage{relsize}
\usepackage{pifont}
\usepackage{currfile}

\newcommand\wordcount{\input{|"texcount -inc -sum -0 -utf8 -ch -template={SUM} \currfilepath"}}


\begin{document}

% The title is set in 14-point Helvetica bold
\title{To write}

% The rest of the title block is set in 12 points Helvetica
\author {Hao Wen, Jingsu Kang\\ % First name, initials and surnames, no ``and''
\ \\ % leave an empty line between authors and affiliation
China Agricultural University\\  % gives affiliation of the first author only
Beijing, China} % city, [state or province,] country only

\maketitle

%%%%%%%%%%%%%%%%%%%%%%%%%%%%%%%%%%%%%%%%%%%%%%%%%%%%%%%%%%
% NOTE: The body of the abstract (exclusive of the title, authors, and authors’ affiliations) can be up to 300 words at most
%%%%%%%%%%%%%%%%%%%%%%%%%%%%%%%%%%%%%%%%%%%%%%%%%%%%%%%%%%


Aim: to write

Methods: to write

Results: to write

Conclusion: to write


% \ifhandout
% % do nothing
% \else
% \begin{center}
% {\larger[2]\color{red} \ding{43}\ding{43}  Totally \wordcount words \reflectbox{\ding{43}\ding{43}}}
% \end{center}
% \fi


\end{document}
